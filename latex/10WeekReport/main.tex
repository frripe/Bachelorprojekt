\documentclass[10pt,a4paper]{article}
\usepackage{amsmath,graphicx}
\usepackage[utf8]{inputenc}
\usepackage{gantt}
\usepackage{tikz}
\usepackage{float}
\usepackage{longtable}
\usepackage{xcolor}
\usepackage{hyperref}
\usepackage{siunitx}
\usepackage{caption}
\usepackage{subcaption}
\usepackage[
backend=biber,
style=alphabetic,
sorting=ynt
]{biblatex}
\addbibresource{references.bib}


\title{{Bachelor's project\\[0.5em]}
       {\bf \huge Image-based quality evaluation of otoscopy images\\[0.5em]}
       {\bf Weekly Report 10}}
\author{Freja Rindel Peulicke, s185393}
\date{\today}

\setlength{\parindent}{0mm}
\setlength{\parskip}{\medskipamount}

\begin{document}
\maketitle

\section*{Literature}
%\textit{\textbf{Hint:}Try to write this section so it can be used directly in your Previous Work chapter in your thesis.}

\section*{What has been done this week}
%\textit{\textbf{Hint:}Try to write this section so it can be used directly in your thesis. Also use drawings and figures.}


\newpage







\newpage
\section*{Project status according to the study plan}
Many errors have been corrected in the data set and combination of the metrics. A method for choosing the best metric and threshold has been described. Next step is to implement the metric and threshold choosing system and then test it on the test data set.

% http://www.martin-kumm.de/wiki/doku.php?id=05Misc%3AA_LaTeX_package_for_gantt_plots
\section*{Overall Project Plan}
\hspace{-4.3cm}
\begin{gantt}{14}{12}
    \begin{ganttitle}
        \titleelement{September}{5}
        \titleelement{October}{4}
        \titleelement{November}{3}
    \end{ganttitle}
    \begin{ganttitle}
        \numtitle{1}{1}{12}{1}
    \end{ganttitle}
    \ganttbar{\textbf{Activity $|$ (Risk)}}{0}{0}
    \ganttbar[pattern=crosshatch, color=green]{Write introduction and project plan $|$ (1)}{0}{2}
    \ganttbar[pattern=crosshatch, color=green]{Select sharpness-metrics to include $|$ (2)}{2}{2}
    \ganttbarcon[pattern=crosshatch, color=orange]{Find and run code, test it + write $|$ (3)}{3}{2}
    \ganttmilestone[color=red]{Hand in project plan}{4.5}
    \ganttbar{Implement and test program with GUI + write $|$ (2)}{5}{3}
    \ganttbar{Research on active contours + write $|$ (2)}{8}{1}
    \ganttbarcon{Implement and test active contours + write $|$ (3)}{9}{1}
    \ganttbar{Research to find a solution to earwax + write $|$ (2)}{10}{1}
    \ganttbarcon{Implement and test earwax-solution + write $|$ (3)}{11}{1}
    \ganttbar{Read through + small corrections $|$ (1)}{10}{2}
    \ganttmilestone[color=red]{Hand in}{12}
\end{gantt}


\section*{Plan for the next weeks}

\begin{enumerate}
\item Implement choosing metric
\item Test the metrics on test data set (the images not used for training) and report results
\item How are images transferred from otoscopy to pc?
\item Start designing and implementing the final program
\end{enumerate}



\section*{References}
\printbibliography[type=online,title={Code downloads}]
\printbibliography[type=book,title={Books}]
\printbibliography[type=article,title={Articles}]


\end{document}